% Tipo
\documentclass{article}

% Márgen
\usepackage[margin = 1.5cm]{geometry}

% Símbolos
\usepackage{amsmath}

% Controlar tablas
\usepackage{float}

% Mapas de Karnaugh
\usepackage{karnaugh-map}

\begin{document}
    \title{
        Organización y Arquitectura de Computadoras \\
        2019-2 \\
        Práctica 3: Circuitos Combinacionales
    }
    \author{
        Sandra del Mar Soto Corderi \\
        Edgar Quiroz Castañeda
    }
    \date{
        3 de marzo del 2019
    }

    \maketitle

    \section{Ejercicios}

    \begin{enumerate}
        \item {
            Desarrolla un circuito que simule el comportamiento de la implicación
            lógica. Sólo puedes hacer uso de fuentes de alimentación power y 
            ground, transistores tipo PNP y NPN y pines de entrada y salida. \\
            Como es un operador binario de valores lógicos, entonces las entradas
            son dos variables lógicas $\in \{1, 0\}^2$ y la salida es otro valor lógico
            $\in \{1, 0\}$.\\
            Luego, la implicación se defina como sigue
            \begin{table}[H]
                \caption{Tabla de verdad de la implicación}
                \begin{center}
                    \begin{tabular}{|l|l|l|}
                        \hline
                        $p$ & $q$ & $p \implies q$ \\ \hline
                        0   & 0   & 1              \\ \hline
                        0   & 1   & 1              \\ \hline
                        1   & 0   & 0              \\ \hline
                        1   & 1   & 1              \\ \hline
                    \end{tabular}
                \end{center}
            \end{table}
            Y el mapa de Karnaugh correspondiente es 

            \begin{center}
                \begin{karnaugh-map}[2][2][1][$p$][$q$]
                    \minterms{0,2,3}
                    \maxterms{1}
                    \implicant{2}{3}
                    \implicant{0}{2}
                \end{karnaugh-map}
            \end{center}

            Por lo que la regla de correspondencia es 
            \[(p \implies q) = p' + q\]
            
        }
        \item {
            Sean $x, y \in \{0, 1, 2, 3\}$. Desarrolla un comparador electrónico d
            e 2 bits, las salidas del comparador deben ser
            \begin{itemize}
                \item {
                    $x<y$
                }
                \item {
                    $x=y$
                }
                \item {
                    $x>y$
                }
            \end{itemize}
            Podemos considerar los números del 0 al 3 en su representación 
            binaria. Como sólo son dos dígitos, entonces cada número puede ser 
            representado con dos variables lógicas, lo que da un total de 4 
            variables lógicas, una para cada bit de cada número. \\
            Y a su vez, hay tres salidas, una por cada relación que hay que 
            modelar.

            \begin{table}[H]
                \caption{Tabla de verdad de las relaciones anteriores}
                \begin{center}
                    \begin{tabular}{|l|l|l|l|l|l|l|}
                        \hline
                        $a$ & $b$ & $c$ & $d$ & $ab<cd$ & $ab=cd$ & $ab>cd$ \\ \hline
                        0   & 0   & 0   & 0   & 0       & 1       & 0       \\ \hline
                        0   & 0   & 0   & 1   & 1       & 0       & 0       \\ \hline
                        0   & 0   & 1   & 0   & 1       & 0       & 0       \\ \hline
                        0   & 0   & 1   & 1   & 1       & 0       & 0       \\ \hline
                        0   & 1   & 0   & 0   & 0       & 0       & 1       \\ \hline
                        0   & 1   & 0   & 1   & 0       & 1       & 0       \\ \hline
                        0   & 1   & 1   & 0   & 1       & 0       & 0       \\ \hline
                        0   & 1   & 1   & 1   & 1       & 0       & 0       \\ \hline
                        1   & 0   & 0   & 0   & 0       & 0       & 1       \\ \hline
                        1   & 0   & 0   & 1   & 0       & 0       & 1       \\ \hline
                        1   & 0   & 1   & 0   & 0       & 1       & 0       \\ \hline
                        1   & 0   & 1   & 1   & 1       & 0       & 0       \\ \hline
                        1   & 1   & 0   & 0   & 0       & 0       & 1       \\ \hline
                        1   & 1   & 0   & 1   & 0       & 0       & 1       \\ \hline
                        1   & 1   & 1   & 0   & 0       & 0       & 1       \\ \hline
                        1   & 1   & 1   & 1   & 0       & 1       & 0       \\ \hline
                    \end{tabular}
                \end{center}
            \end{table}
            Los mapas de Karnaugh correspondientes son
            \begin{itemize}
                \item {
                    Igualdad 
                    \begin{center}
                        \begin{karnaugh-map}[4][4][1][$ab$][$cd$]
                            \minterms{0, 5, 10, 15}
                            \maxterms{1, 2, 3, 4, 6, 7, 8, 9, 11, 12, 13, 14}
                            \implicant{0}{0}
                            \implicant{5}{5}
                            \implicant{10}{10}
                            \implicant{15}{15}
                        \end{karnaugh-map}
                    \end{center}
                    Por lo que la regla de correspondencia es 
                    \[(ab = cd) = a'b'c'd' + a'bc'd + abcd + ab'cd'\]
                }
                \item {
                    Menor que
                    \begin{center}
                        \begin{karnaugh-map}[4][4][1][$ab$][$cd$]
                            \minterms{1, 2, 3, 6, 7, 11}
                            \maxterms{0, 4, 5, 8, 9, 10, 12, 13, 14, 15}
                            \implicant{3}{6}
                            \implicant{1}{3}
                            \implicantedge{3}{3}{11}{11}
                        \end{karnaugh-map}
                    \end{center}
                    Por lo que la regla de correspondencia es 
                    \[(ab < cd) = ac' + bc'd' + abd' \]
                }
                \item {
                    Mayor que
                    \begin{center}
                        \begin{karnaugh-map}[4][4][1][$ab$][$cd$]
                            \minterms{4, 8, 9, 12, 13, 14}
                            \maxterms{0, 1, 2, 3, 5, 6, 7, 10, 11, 15}
                            \implicant{12}{9}
                            \implicant{4}{12}
                            \implicantedge{12}{12}{14}{14}
                        \end{karnaugh-map}
                    \end{center}
                    Por lo que la regla de correspondencia es 
                    \[(ab < cd) = a'c + b'cd + a'b'd \]
                }
            \end{itemize}
        }
    \end{enumerate}

    \section{Preguntas}

    \begin{enumerate}
        \item {
            ¿Cuál es el procedimiento a seguir para desarrollar un circuito que 
            recuelva un problema que involucre lógica combinacional?
            \begin{enumerate}
                \item {
                    Identificar las variables lógicas de entrada y las salidas 
                    lógicas.
                }
                \item {
                    Construir la tabla de verdad del circuito.
                }
                \item {
                    Minimizar la expresión que representa la tabla de verdad. 
                    Esto se puede hacer usando manipulación algebraica, mapas de 
                    Karnaugh, el algoritmo de Quine-McCluskey o el método de 
                    Expreso.
                }
                \item {
                    Diseñar o especificar el circuito correspondiente a la 
                    expresión minimizada.
                }
            \end{enumerate}
        }
        \item {
            Si una función de conmutación se evalua a más ceros que unos, ¿es
            conveniento usar mintérminos o maxtérminos? ¿Y en caso contrario? \\
            Si se toman mintérminos, entonces la cantidad de términos en la 
            expresión es la misma que la cantidad de casos donde la función es 1.
            Si se usan maxtérminos, entonces esta cantidad es la misma que la 
            cantidad de casos donde la función es 0. \\
            Por lo que si hay más unos, la expresión quedará más corta si se 
            usan maxtérminos, y si hay más ceros, quedará más corta usando 
            mintérminos. \\
            Aunque si se minimiza la expresión, se debería obtener algo más o 
            menos del mismo tamaño sin importar si se inició con mintérminos o 
            maxtérminos.
        }
        \item {
            En base al trabajo realizado, ¿cuáles son los inconvenientes de 
            desarrollo de circuitos de forma manual? \\
            El proceso de representación por expresiones y el proceso de 
            minimización es tedioso, y la complejidad de los patrones y longitud
            de las expresiones aumenta exponensialmente con el número de 
            variables. Entonces, hay un punto en el que no es viable hacerlo a 
            mano.
        }
    \end{enumerate}
\end{document}