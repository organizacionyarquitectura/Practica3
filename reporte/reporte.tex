%Tipo
\documentclass{article}

\begin{document}
    \title{
        Organización y Arquitectura de Computadoras \\
        2019-2 \\
        Práctica 3: Circuitos Combinacionales
    }
    \author{
        Sandra del Mar Soto Corderi \\
        Edgar Quiroz Castañeda
    }
    \date{
        3 de marzo del 2019
    }

    \maketitle

    \section{Ejercicios}

    \begin{enumerate}
        \item {
            Desarrolla un circuito que simule el comportamiento de la implicación
            lógica. Sólo puedes hacer uso de fuentes de alimentación power y 
            ground, transistores tipo PNP y NPN y pines de entrada y salida. 
        }
        \item {
            Sean $x, y \in \{1, 2, 3\}$. Desarrolla un comparador electrónico d
            e 2 bits, las salidas del comparador deben ser
            \begin{itemize}
                \item {
                    $x<y$
                }
                \item {
                    $x=y$
                }
                \item {
                    $x>y$
                }
            \end{itemize}
        }
    \end{enumerate}

    \section{Preguntas}

    \begin{enumerate}
        \item {
            ¿Cuál es el procedimiento a seguir para desarrollar un circuito que 
            recuelva un problema que involucre lógica combinacional?
        }
        \item {
            Si una función de conmutación se evalua a más ceros que unos, ¿es
            conveniento usar mintérminos o maxtérminos? ¿Y en caso contrario?
        }
        \item {
            En base al trabajo realizado, ¿cuáles son los inconvenientes de 
            desarrollo de circuitos de forma manual?
        }
    \end{enumerate}
\end{document}